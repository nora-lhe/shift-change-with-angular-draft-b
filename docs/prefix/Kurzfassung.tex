\section*{Kurzfassung}
Diese Studienarbeit behandelt die Entwicklung einer Anwendung zum Verwalten von Schichtwechseln unter der Verwendung des Google Design Sprints. Zunächst wurden ein UX- und Benutzerkonzept entwickelt, um die Nutzerfreundlichkeit und Effektivität der Anwendung zu gewährleisten. Im Anschluss daran wurde eine Nutzerumfrage durchgeführt, um Feedback von den Anwendern zu sammeln. Diese Umfrage zielte darauf ab, herauszufinden, wie das Design bei den Nutzern ankommt, ob die Anwendung intuitiv ist und welche Verbesserungsvorschläge oder zusätzlichen Informationen von den Nutzern gewünscht werden. Die Ergebnisse der Umfrage wurden sorgfältig ausgewertet. Es stellte sich heraus, dass die Nutzer das Design im Allgemeinen gut fanden, jedoch einzeln Verbesserungsvorschläge und der Wunsch nach zusätzlichen Informationen geäußert wurde. Basierend auf diesen Erkenntnissen wurde das Benutzerkonzept überarbeitet, um besser auf die Bedürfnisse und Wünsche der Nutzer einzugehen. In der letzten Phase der Entwicklung wurde anhand des überarbeiteten Benutzerkonzepts ein Prototyp mit Angular erstellt. Diese dient als Grundlage für die weitere Entwicklung und Verfeinerung der Anwendung, um eine optimale Nutzererfahrung zu gewährleisten.